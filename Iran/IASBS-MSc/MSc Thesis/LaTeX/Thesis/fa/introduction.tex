\newpage
\vspace*{-1cm}
\section*{مقدمه}
\addcontentsline{toc}{section}{پیش‌گفتار}

مقدار سنجي عدم قطعيت [1] در فيزيک و مهندسي اغلب شامل مطالعه معادلات ديفرانسيل با مشتقات جزئي با ميدان ضرايب تصادفي است. براي درک رفتار يک سيستم شامل عدم قطعيت، ميتوان کميت‌هاي فيزيکي مشتق شده از معادلات ديفرانسيل را به عنوان توابعي از ميدان ضرايب استخراج کرد؛ اما حتي با گسسته‌سازي مناسب روي دامنه معادله و برد متغييرهاي تصادفي، اين کار به طور ضمني به حل عددي معادله ديفرانسيل با مشتق جزئي به تعداد نمايي مي‌انجامد.

يکي از روش‌هاي متداول براي مقدار سنجي عدم قطعيت روش نمونه‌ برداري مونته کارلو است. گرچه اين روش در بسياري از موارد کاربردي است اما کميت اندازه‌گيري شده ذاتا داراي پراش است. به‌علاوه اين روش قادر به پيدا کردن جواب‌هاي جديد در صورتي که قبلا نمونه‌گيري نشده باشند، نيست.

روش گالرکين تصادفي [3] [4] با استفاده چند جمله‌اي‌هاي آشوب [5] [6] يک جواب تصادفي را روي فضاي متغييرهاي تصادفي بسط ميدهد و به اين طريق مسئله با بعد بالا را به تعدادي معادله ديفرانسيل با مشتقات جزئي معين تبديل مي‌کند. اين گونه روش‌ها به دقت زيادي درباره تعيين توزيع عدم قطعيت نيازمند هستند و از آن‌جا که پايه‌هاي استفاده شده مستقل از مسئله هستند، وقتي بعد متغييرهاي تصادفي بالا باشد هزينه محاسباتي بسيار زياد خواهد شد.

تذکر مي‌دهيم که کار ما با [7] [8] [9] [10] [11] که يک معادله ديفرانسيل با مشتقات جزئي را با شبکه‌هاي عصبي حل ميکنند، متفاوت است. هدف کار ما پارامتري کردن جواب يک معادله ديفرانسيل معيين به کمک شبکه‌هاي عصبي و سپس استفاده از روش‌هاي بهينه‌سازي براي يافتن جواب معادله است. همچنين کار ما با [12] که در آن يک معادله ديفرانسيل با مشتقات جزئي به عنوان يک مسئله کنترل آشوب به کمک شبکه‌هاي عصبي حل شده نيز متفاوت است. در کار ما تابع مورد نظر براي پارامتري‌سازي روي ميدان ضرايب معادله ديفرانسيل با مشتقات جزئي تعريف شده است. در واقع مطابق اطلاعات ما، تاکنون کاهش بعد مبتني بر نمايش شبکه‌عصبي براي حل معادلات ديفرانسيل با مشتقات جزئي همراه با عدم قطيت استفاده نشده است.

\section*{انگیزه و هدف}


\section*{روش‌های تقریبی }

\section*{تعریف مسئله }

\section*{شبکه‌های عصبی }

\section*{روش پیشنهادی }

\section*{نتیجه‌گیری و کارهای پیش‌رو}
