برای مدل‌سازی پدیده‌های واقعی با معادلات دیفرانسیل با مشتقات جزئی که شامل عدم قطعیت است، یکی از مشکلات وجود مجموعه‌ای از پدیده‌ها است که به عنوان مشکلات ابعاد بالا شناخته می‌شوند. خوشبختانه، اغلب تغییرات متغییرهای مدل می‌توانند توسط تعداد کمی خصوصیات دامنه توسط روش‌های کاهش مدل، ثبت شوند. برای مثال، می‌توان با استفاده از روش‌های مبتنی بر شبکه‌های عصبی متغییرهای مورد نظر را به عنوان تابعی از ضرایب ورودی اندازه‌گیری کرد. در این‌صورت، نمایش پذیری متغیرها توسط چنین شبکه‌ای را می‌توان با دید شبکه عصبی به عنوان یک تحول زمانی برای پیداکردن جواب‌های مدل توجیه کرد. در این پایان نامه، ما یک روش میانبر برای پیدا کردن جواب‌های مدل روی دو معادله دیفرانسیل با مشتقات جزئی معروف در فیزیک و مهندسی را بازبینی مینمائیم. همچنین، ما به سراغ بررسی یک روش عددی سنتی از نظر تئوری خواهیم رفت و از این طریق، احتمالات جدیدی برای استفاده از شبکه‌های عصبی در حل معادلات دیفرانسیل را مطرح خواهیم نمود.
%مدلسازي بيشتر مسايل فيزيک به معادلات ديفرانسيل همراه با ضرايب عدم قطعيت منتهي مي‌شوند، به عبارت ديگر معادلاتي با ضرايب تصادفي. در بيشتر موارد اين عدم قطعيت مشتق شده از معادلات ديفرانسيل را مي‌توان با تعدادي از ويژگي‌هاي ميدان ضرايب کنترل کرد. بر همين اساس، پيشنهاد ما استفاده از يک روش بر پايه شبکه‌هاي عصبي براي پارامتري کردن کميت‌هاي فيزيکي مورد نظر به عنوان تابعي از ضرايب ورودي است.
%نمايش کميت مورد نظر با استفاده از شبکه عصبي مي‌تواند با ديد شبکه عصبي به عنوان تکامل دهنده در طي زمان براي پيدا کردن جواب‌هاي معادله ديفرانسيل با مشتقات جزئي تعديل شود. 

%دو معادله ديفرانسيل با کاربرد گسترده در فيزيک و مهندسي را مورد بررسي قرار مي‌دهيم که عبارت‌اند از معادله لاپلاس و معادله شرودينگر غيرخطي.
%معادلات بيضوي عموماً براي مطالعه‌ي اعمال گرماي يکنواخت روي يک ماده دلخواه به کار مي‌روند. زماني که ماده ناهمگن باشد(که با استفاده از ضرايب تصادفي در معادله بيضوي مدل مي‌شوند)، معمولاً به دنبال يافتن انتقال دماي موثر ماده هستيم.

%معادله شرودينگر غيرخطي براي يافتن انتشار نور در موج‌بر و همچنين در پديده‌هاي مکانيک کوانتومي که ذرات بوزونيک در پايين‌ترين حالت انرژي متمرکز شده‌اند (چگالش بوز-انيشتين) به کار مي‌رود. ما به دنبال يافتن جواب اين پرسش هستيم که انرژي چنين حالت اساسي‌اي وقتي معادله شرودينگر غيرخطي در مورد ميدان پتانسيل تصادفي مطرح شود چگونه رفتار خواهد کرد.
%براي سادگي در هر دو معادله ديفرانسيل، شرط مرزي، به طور متناوب در نظر گرفته شده است. مباحث عددي مورد نظر در اين پايان نامه مبتني بر روش تفاضل متناهي و روش شبکه هاي عصبي است.
%مدلسازي بيشتر مسائل فيزيک به معادلات ديفرانسيل همراه با ضرايب عدم قطعيت به عبارت ديگر معادلاتي با ضرايب تصادفي منتهي مي‌شوند.  در بيشتر موارد اين عدم قطعيت مشتق شده از معادلات ديفرانسيل را مي‌توان با تعدادي از ويژگي‌هاي ميدان ضرايب کنترل کرد. بر همين اساس، پيشنهاد ما استفاده از يک روش بر پايه شبکه‌هاي عصبي براي پارامتري کردن کميت‌هاي فيزيکي مورد نظر به عنوان تابعي از ضرايب ورودي است. نمايش کميت مورد نظر با استفاده از شبکه عصبي مي‌تواند با ديد شبکه عصبي به عنوان تکامل دهنده در طي زمان براي پيدا کردن جواب‌هاي معادله ديفرانسيل با مشتقات جزئي تعديل شود. 
%دو معادله ديفرانسيل با کاربرد گسترده در فيزيک و مهندسي را مورد بررسي قرار مي‌دهيم که عبارت‌اند از معادله لاپلاس و معادله شرودينگر غيرخطي.\\
%معادلات بيضوي عموماً براي مطالعه‌ي اعمال گرماي يکنواخت روي يک ماده دلخواه به کار مي‌روند. زماني که ماده ناهمگن باشد(که با استفاده از ضرايب تصادفي در معادله بيضوي مدل مي‌شوند)، معمولاً به دنبال يافتن انتقال دماي مؤثر ماده هستيم.
%معادله شرودينگر غيرخطي براي يافتن انتشار نور در موج‌بر و همچنين در پديده‌هاي مکانيک کوانتومي که ذرات بوزونيک در پايين‌ترين حالت انرژي متمرکز شده‌اند (چگالش بوز-انيشتين) به کار مي‌رود. ما به دنبال يافتن جواب اين پرسش هستيم که انرژي چنين حالت اساسي‌اي وقتي معادله شرودينگر غيرخطي در مورد ميدان پتانسيل تصادفي مطرح شود چگونه رفتار خواهد کرد.
%براي سادگي در هر دو معادله ديفرانسيل، شرط مرزي، به طور متناوب در نظر گرفته شده است. مباحث عددي مورد نظر در اين پايان نامه مبتني بر روش تفاضل متناهي و روش شبکه‌هاي عصبي است.
\keywords{شبکه‌های عصبی، روش تفاصلات متناهی، روش المان‌های متناهی، معادلات دیفرانسیل با مشتقات جزئی، عدم قطعیت.}
