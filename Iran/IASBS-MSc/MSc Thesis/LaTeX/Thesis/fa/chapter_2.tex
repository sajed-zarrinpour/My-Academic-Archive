\chapter{معرفی دستور زبان و عملگرها}
در این فصل به معرفی دستورات اصلی زبان می‌پردازیم. در بخش اول انواع داده و دستورات ابتدایی زبان را مشخص می‌کنیم. در بخش دوم تعریف توابع را بررسی می کنیم. در انتها بخش سوم به آرایه‌ها و ماتریس‌ها اختصاص یافته است.
\section{انواع داده و دستورات ابتدایی}
نرم افزار FreeFem++ به ذات یک زبان تفسیری است؛ از نوع زبان های وابسته به انواع و چند ریختی است و به پردازش خطا مجهز است. هر متغییر باید با یک نوع مشخص تعریف شود. و همانند C++ هر دستور باید به کاراکتر ";" ختم شود. نوشتن توضیح نیز به طور مشابه با کاراکتر های "//" و "/**/" به ترتیب برای توضیحات یک و چند خطی استفاده می شوند. برای قطع روند اجرا از دستور exit(0) استفاده می کنیم.\\
متغیر verbosity سطح چاپ را تنظیم می کند و مقدار پیش فرض آن 2 است. انواع داده پایه ای عبارت اند از :
\begin{enumerate}
	\item مختصات فعلی x, y و z
	\item عملگر های پیش فرض دیفرانسیل:
		  $dx=\frac{\partial}{\partial x}$ ،
		  $dy=\frac{\partial}{\partial y}$ ،
		  $dz=\frac{\partial}{\partial z}$ ،
		  $dxx=\frac{\partial}{\partial xx}$ ،
		  $dyy=\frac{\partial}{\partial yy}$ ،
		  $dzz=\frac{\partial}{\partial zz}$ ،
		  $dxy=\frac{\partial}{\partial xy}$ ،
		  $dxz=\frac{\partial}{\partial xz}$ و
		  $dyz=\frac{\partial}{\partial yz}$ .
	\item عدد صحیح int
	\item عدد حقیقی real
	\item عدد مختلط complex
	\item رشته string
	\item آرایه با عناصر حقیقی real[int]
	\item آرایه با عناصر مختلط complex[int]
	\item ماتریس با درایه های حقیقی real[int,int]
	\item ماتریس با درایه های مختلط complex[int,int]
	\item مثلث بندی دو بعدی mesh
	\item مثلث بندی سه بعدی mesh3
	\item فضای عناصر متناهی fespace
	\item انتگرال یک بعدی int1d
	\item انتگرال دو بعدی int2d
	\item انتگرال سه بعدی int3d
\end{enumerate}
\section{تعریف توابع}
توابع می توانند برای مصارف زیر تعریف شوند :

\begin{enumerate}
	\item تعریف توابع تحلیلی
		 \begin{LTR}
			 \begin{lstlisting}
 func u0=1.*(x>=2 & x<=2)  
		   \end{lstlisting}
		 \end{LTR}
	\item تعریف یک تابع یا آرایه در فضای عناصر متناهی
		\begin{LTR}
			\begin{lstlisting}
 Vh u=exp(-x^2-y^2)  
			\end{lstlisting}
		\end{LTR}
		لازم به ذکر است که در این مورد u یک تابع در فضای عناصر متناهی است. بنابراین u[] مقدار u در تمام درجات آزادی را بر می گرداند. همچنین اگر مقدار تابع در i-امین درجه آزادی مد نظر باشد می توانیم از دستور u[][i] استفاده کنیم و نهایتا برای یافتن مقدار u در نقطه دلخواه (a,b) از دستور u(a,b) استفاده می کنیم.
	\item تعریف توابع معمولی
	\begin{LTR}
		\begin{lstlisting}
 func real f ( int a , real [ int ] U ) {
 	Vh NU ;
 	NU []= U ;
 	return a * NU ;
 }
 Vh U =x, FNU =f (5 , U []) ;
  
		\end{lstlisting}
	\end{LTR}
	\item تعریف ماکرو ها
	\begin{LTR}
	\begin{lstlisting}
 macro Grad ( u ) [dx( u ) ,dy( u ) ]// in 2D
 macro Grad ( u ) [dx( u ) ,dy( u ) ,dz( u) ]// in 3D
 macro div (u , v ) [dx( u )+dy( v ) ]// in 2D
 macro div (u ,v , w ) [dx(u ) +dy( v ) +dz( w ) ]// in 3D

  
		\end{lstlisting}
	\end{LTR}
دقت کنید که دستور تعریف ماکرو به جای کاراکتر ";" به کاراکتر "//" ختم می شود.	
\end{enumerate}

\section{آرایه ها و ماتریس ها}
مانند متلب آرایه ها را می توان به صورت  U=start:step:end real[int] تعریف کرد و با دستور U(i) می توان به i-امین عنصر آرایه دسترسی یافت. تعریف و دسترسی به درایه های ماتریس ها نیز به طور مشابه انجام می شود. تعدادی از دستورات مربوط به ماتریس ها و آرایه ها به شرح زیر است.
\begin{LTR}
	\begin{lstlisting}
 real [int] u1 =[1 ,2 ,3] , u2 =2:4; // defining u1 and u2
 real u1pu2 = u1 ’* u2 ; // give the scalar product of u1 and u2 , here u1 ’ is the transpose of u1;
 real [int] u1du2 = u1 ./ u2 ; // divided term by term
 real [int] u1mu2 = u1 .* u2 ; // multiplied term by term
 matrix A = u1 * u2 ’; // product of u1 and the transpose of u2
 matrix < complex > C =[ [1 ,1i ] ,[1+2i ,.5*1 i] ];
 real trA = trace ([1 ,2 ,3]*[2 ,3 ,4] ’) ; // trace of the matrix
 real detA = det ([ [1 ,2] ,[ -2 ,1] ]) ; // just for matrix 1x1 and 2x2
	\end{lstlisting}
\end{LTR}
 در انتها ذکر این نکته ضروری است که حلقه های تکرار و دستورات ورود و خروج جریان داده همان دستورات زبان C++ هستند. یعنی
 \begin{LTR}
 	\begin{lstlisting}
 for ( INITIALIZATION ; CONDITION ; CHANGE )
 { 
 	BLOCK of calculations 
 }
 	
 while ( CONDITION ) 
 {
 	BLOCK of calculations or change of control variables
 }
 
 int i;
 cout << " std -out" << endl ;
 cout << " enter i= ? ";
 cin >> i ;
 Vh uh =x+y;
 ofstream f (" toto . txt "); f << uh []; // to save the solution
 ifstream f (" toto . txt "); f >> uh []; // to read the solution
 	\end{lstlisting}
 \end{LTR}










