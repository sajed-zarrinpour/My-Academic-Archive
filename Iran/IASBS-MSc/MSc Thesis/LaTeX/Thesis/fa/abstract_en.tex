\noindent 
Towards modeling the real-world phenomenons with partial differential equations that involve uncertainties, one of the major difficulties that show itself is the curse of dimensionality. We encounter that when we was going to model soft tissue deformation using finite element method. Luckily, very often the variability of physical quantities derived from the model can be captured by a few features on the coefficient fields. On the other hand, nowadays, it had seen that the marriage between neural-networks and numerical methods is very fruitful. For example, one can use neural-networks based methods to parametrize the physical quantity of interest as a function of input coefficients; In that case, the representability of such quantity can be justified by viewing the neural-networks as performing time evolution to find the solution to the model, as indeed, in this thesis we would like to study the use of neural-networks on two notable partial differential equations in engineering and physics to that end. Furthermore, we would like to try to find an answer to this question: How simplicity and accuracy are tradings when we use neural-networks?

\latinkeywords{Partial Differential Equations,Finite Element Method, Uncertainty Quantification}
