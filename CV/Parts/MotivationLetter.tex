\begin{cSection}{Motivation Letter}
	\small

I remember the first time I saw a disabled girl who got an artificial hand and picked up a cup with it in a documentary show. I was sixteen and I felt fantastic. From that moment, I knew I would be a researcher.

At the end of my first year in my master's, I've found the paper \href{https://pubmed.ncbi.nlm.nih.gov/28982035/}{\textcolor{blue}{A finite element-based machine learning approach for modeling the mechanical behavior of the breast tissues under compression in real-time}}. To improve it, I decided to incorporate patient-specific information into the model. However, that required me to have a firm background in mathematics and machine learning. Thus, I decided to work on the foundation needed in my master's.

In the winter school \textit{From Biophysical Modeling to Simulation Codes, 2019}, I had acquired a friend from Graz university who was working on modeling the heart. I've found her work fascinating, and it pushed me towards mathematical modeling itself. The following year, I investigated the combination of the finite element method and machine learning to solve PDEs with uncertainties under the supervision of Dr. K. Nedaiasl and the advisory of Dr. P. Razzaghi. Afterward, I tried to deepen my knowledge about inverse problems by participating in online events like `One World IMAGINE Seminars' and `UC Berkeley Applied Math Seminar: Solving inverse problems with deep learning'. Later, I decide to develop mathematical modeling skills through studying advanced melanoma models by study \href{https://www.sciencedirect.com/science/article/abs/pii/S0022519318305903}{\textcolor{blue}{Spatio-Genetic and phenotypic modelling elucidates resistance and re-sensitisation to treatment in heterogeneous melanoma}} in depth. 

During my masters,  I did studied a few machine learning models including \href{https://link.springer.com/article/10.1007/s40304-018-0127-z#:~:text=We%20propose%20a%20deep%20learning,work%20in%20rather%20high%20dimensions.}{\textcolor{blue}{ the deep Ritz method}}, \href{https://www.semanticscholar.org/paper/Solving-PDE-problems-with-uncertainty-using-Khoo-Lu/ffacc17153bf122cc8a726adca6b468cd5fecb54}{\textcolor{blue}{surrogate forward mapping}}, \href{https://maziarraissi.github.io/research/1_physics_informed_neural_networks/}{\textcolor{blue}{physics informed neural networks}} which gave me a deeper understanding of neural networks.
	
	I like biology and as my research background suggests, I am willing to engage complex problems. 
	
	In the project description, it says `A specific difficulty is that the 3D structure recovery involves a large number of nuisance parameters. Data is also very noisy and the 3D molecular structure can be flexible. Hence, using the physics-informed deep learning methods in this setting raises both theoretical and algorithmic challenges '.  I did work on uncertain parameters in my masters and I also studied PINN. Though the project description did not mention whether it is a supervised or unsupervised network it aims but knowing it is about improving cryogenic electron microscopy, I guess it is a supervised learning task. Hence, my first naive idea is to pair up PINN with a surrogate model in a multi-task network. 
	
	My general interest is in using machine learning as a solver for biological PDE models. I like to tackle real problems in biology. I want to see the result of my work in action. I have basic knowledge of mathematical modelling, numerical analysis, uncertainty quantifications. I have outstanding programming skills and experience In general and specifically am working on machine learning since 2018. I am active and creative. Hence, I think I am suited for your project.
	\normalfont	
\end{cSection}
%I read the application announcement. Though speech coding strategies or the CI speech processor were not in my research area, I was working on skills that I think can be of use here; mathematical modeling, inverse problems, model reduction, and the use of neural networks to process surrogate mappings. Hence, I think I can use that foundation in this project.

%I was not motivated to study back in my bachelor's degree. It was till I had a course with an exceptional professor, Dr. H. Saberi Najafi who thaught me a purpose. To understand the real world and find its beuty. Now, I am quite motivated. To me, mathematics is about explaining the real world through a universal language. In my opinion, the best use of mathematics is helping people. Hence, I have an outstanding reason to carry on my research.