
%\begin{cSection}{Statement of research}
%	
%	\years{Subject: }{\scriptsize SOR for ERC-funded PhD scholarship
%		Announcement}\\



%	\vspace*{1.5em}	
%	I want to express my interest in joining your research group. Though all three subjects are fascinating and I think I can do good in them, in particular, I am more interested in Temporal networks. You can find my research background and activities in detail, below.\\
%	It began in 2018 with a single idea:\\
%	``Can we use machine learning techniques to enhance solutions obtained from the finite element method?'' \\
%	This question opened a long way in my research career. Indeed, it has fantastic applications e.g. \href{https://www.ncbi.nlm.nih.gov/pmc/articles/PMC5967816/}{\textcolor{blue}{Concussion classification via deep learning using whole-brain white matter fiber strains}}, \href{https://link.springer.com/chapter/10.1007/3-540-45468-3_189}{\textcolor{blue}{Methods for modeling and predicting mechanical deformations of the breast under external perturbations}}, but I choose to study soft tissue deformations (breast deformations under comparison in particular) in memories of my passed aunt.\\
%	I begin by reading literature in the field. I had found that different medical imaging methods exist for breast imaging, which uses different technologies and produces different results (e.g. a lesion may consider healthy in one method and unhealthy in another). The major contributions that I had found were to combine these imaging strategies to get a single model that can be used to determine the lesion, its location, and its boundaries. In all those papers, one thing was in common: their models were not patient-specific. Hence, I decided to do it patient specific. To this end, I needed to know how to solve PDEs with uncertain coefficients to incorporate patient-specific data, such as elasticity of the body. Thus, my master thesis is entitled `Solving Partial Differential Equations with Uncertainty using Neural Networks'.\\
%	My main interest was to learn mathematical modeling and solving the model numerically. Participating in Iranian-Austrian winter school (2019-details in CV) provided me with a good understanding of mathematical modeling. Now I had a way to incorporate uncertainties in the deformation model. The next interesting subject was to use neural networks to solve PDE models. I went to the computer science department to find help which led to cooperation between the two remarkable researchers, Dr. K. Nedaiasl (my supervisor) and Dr. P. Razaghi (my advisor) from two departments on the matter. I studied different approaches in the literature, such as Deep Ritz Method, Physical Informed Neural Networks, surrogate forward mapping, etc under the supervision of them. I also implement those methods using Python Keras and explore different settings, data normalization methods, and architectures which led to a pre paper that we are currently working on it and we will submit it soon. \\
%	In general, all health care fields related to mathematical modeling is interesting for me. In 2019, I had participated in student presentations at our Institute. There was another participant who was working on Alzheimer's disease. I was interested in her presentation in which, she was working on using stem cells to reproduce lost neurons in neuron loss caused by the disease. The interesting part for me was how the disease develops from amyloid plaques. Later on, I saw a documentary about the process in which the body itself cleans these plaques during the sleep process. I believe it is a very good starting point to study the relationship between the amyloid plaque life cycle and neuron loss progress. Though I  had neither time nor resources to do further research on the matter.\\
%	As a summary of my research background, I had tried to learn mathematical modeling basics and gain further insights into the subject. I also interested to do patient-specific modeling through PDE models with uncertainties at a clinical time using neural networks.\\
%	As for the future, I intended to continue my research on health care fields related to mathematical modeling. Although I didn't choose any specific health care field yet, I developed skills that I believe can be useful in most of them. I will continue my work on the soft tissue deformation until the beginning of my Ph.D. I am eager to work on neuroscience, and I think I can achieve very good results in your foundation.\\	
%	\vspace*{1em}

%	Sajed Zarrinpour Nashroudkoli\\
%	\normalsize
%	---------------------------------------\\
%	\tiny\textit{\scriptsize Guilan, Iran, \today}\normalsize

%\end{cSection}