\begin{cSection}{Sample Scientific writing}
	\years{Subject: }{\scriptsize Abstract of Ms.C. thesis}\\
	\vspace*{1em}
	\small
	Towards modeling the real-world phenomenons with partial differential equations
	that involve uncertainties, one of the major difficulties is a set of phenomena known
	as the curse of dimensionality. Luckily, very often the variability of physical quantities
	derived from the model can be captured by a few features on the coeffcient fields, with
	what so called model reduction techniques. On the other hand, neural networks are
	good at finding hidden maps on the data. For example, one can use neural-networks
	based methods to parametrize the physical quantity of interest as a function of input
	coefficients. In that case, the representability of such quantity can be justified by viewing 
	the neural networks as performing time evolution to find the solution to the model.
	Indeed, in this thesis, we review a surrogate forward neural network model used to solve
	two notable partial differential equations in engineering and physics. Also, we explain
	possibilities that neural networks comes forward throw looking at the mathematical
	analysis of a well known method, namely finite element method.
	\normalsize
	--------------------------------------------------------------------------------\\
	\tiny\textit{
		\scriptsize The full version can be found \href{
			https://github.com/sajed-zarrinpour/My-Academic-Archive/blob/master/Iran/IASBS-MSc/MSc%20Thesis/main.pdf
		}{\textcolor{blue}{here}.
	}}
	\normalsize
\end{cSection}


