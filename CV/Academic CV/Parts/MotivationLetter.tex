\begin{cSection}{Motivation Letter}
	\small

I am writing to apply for a Ph.D. position in Computer simulations of Spatio-temporal processes during inflammation in the Department of Biological Physics at the  Eötvös Loránd University, Hungary. I believe that my skills and experience present an excellent fit for this position. I am enormously interest in your research activities and my background completely matches yours, it will be a great opportunity for me to part of such an outstanding group. So, I would appreciate it if you could
consider my application.

I have completed my master's degree in numerical analysis at the institute for advanced studies in basic sciences. During this program, I was working with the department of mathematics and the department of computer science under the supervisory of Dr. Khadijeh Nedaiasl and the advisory of Dr. Parvin Razaghi. I have investigated three neural models for solving partial differential equations, the deep Ritz method, surrogate forward mapping, and physics informed neural networks, which due to my interest in uncertain parameters I chose the surrogate forward mapping to study in-depth in my thesis.

On account of my interest in computer science, I began by studying the possibility of using neural networks for solving PDE models.  There I came across a paper on using a finite element-based machine learning approach for modeling the mechanical behavior of the breast tissues under compression. I always had a taste for biology that originated in my teens, and this work pushed me towards researching more on mathematical models. In 2019, I had participated in the winter school \textit{From Biophysical Modeling to Simulation Codes}. There, I had acquired a friend from Graz university who was working on modeling the heart. I've found her work fascinating. My research line gets a more precise shape after that school.
I tried to deepen my knowledge about inverse problems by participating in online events. In 2020, I was applying for a Ph.D. position whose supervisor asked me to read a paper entitled \textit{Spatio-Genetic and phenotypic modeling elucidates resistance and re-sensitization to treatment in heterogeneous melanoma} in depth. That was the most challenging model I've ever seen.   To understand it, I did dive deep into biology. Although I understood the model very well I wasn't able to replicate its results at that time. 

To sum up, given my mathematical insight and my computer science experiences, I believe I am an ideal candidate for this Ph.D. program. I am eager and able to learn any kind of required programming languages and courses. I hope to do my Ph.D. in mathematical modeling and later to be a professor engaged in research.



	\normalfont
	
	
	---------------------------------------\\
	\tiny\textit{\scriptsize Guilan, Iran, \today}
	\normalsize	
\end{cSection}
%I read the application announcement. Though speech coding strategies or the CI speech processor were not in my research area, I was working on skills that I think can be of use here; mathematical modeling, inverse problems, model reduction, and the use of neural networks to process surrogate mappings. Hence, I think I can use that foundation in this project.

%I was not motivated to study back in my bachelor's degree. It was till I had a course with an exceptional professor, Dr. H. Saberi Najafi who thaught me a purpose. To understand the real world and find its beuty. Now, I am quite motivated. To me, mathematics is about explaining the real world through a universal language. In my opinion, the best use of mathematics is helping people. Hence, I have an outstanding reason to carry on my research.

%I am Sajed Zarinpour. I finished my master's in applied mathematics on 2020 October 6 at the Zanjan Institute for advanced studies in basic sciences, Iran. I want to express my interest in projects ESR6. Considering my background, I believe I can do well in the project.

%My journey began with reading the paper \href{https://pubmed.ncbi.nlm.nih.gov/28982035/}{\textcolor{blue}{A finite element-based machine learning approach for modeling the mechanical behavior of the breast tissues under compression in real-time}} at the end of my first year in my master's. I thought of incorporating patient-specific information into the model to improve it. However, that required me to have a firm background in mathematics and machine learning. Thus, I decided to work on the foundation needed in my master's. Hence, I tried to dive deep into mathematical modeling by participating in winter school \textit{From Biophysical Modeling to Simulation Codes 2019} first. There, I had acquired a friend from Graz university who was working on modeling the heart. I've found her work fascinating, and it pushed me towards mathematical modeling itself. 

%Next, I investigated the finite element method and machine learning to solve PDEs with uncertainties. I gained a deeper understanding of neural networks by study methods like \href{https://link.springer.com/article/10.1007/s40304-018-0127-z#:~:text=We%20propose%20a%20deep%20learning,work%20in%20rather%20high%20dimensions.}{\textcolor{blue}{ the deep Ritz method}}, \href{https://www.semanticscholar.org/paper/Solving-PDE-problems-with-uncertainty-using-Khoo-Lu/ffacc17153bf122cc8a726adca6b468cd5fecb54}{\textcolor{blue}{surrogate forward mapping}}, \href{https://maziarraissi.github.io/research/1_physics_informed_neural_networks/}{\textcolor{blue}{physics informed neural networks}} under the supervision of Dr. Khadijeh Nedaiasl and the advisory of Dr. Parvin Razzaghi. 
	
%	Afterward, I tried to deepen my knowledge about inverse problems by participating in online events like `One World IMAGINE Seminars' and `UC Berkeley Applied Math Seminar: Solving inverse problems with deep learning.' 
%	Later in 2020, I decided to develop mathematical modeling skills through studying advanced melanoma models by study \href{https://www.sciencedirect.com/science/article/abs/pii/S0022519318305903}{\textcolor{blue}{Spatio-Genetic and phenotypic modelling elucidates resistance and re-sensitisation to treatment in heterogeneous melanoma}} in depth. To understand it, I did dive deep into biology and studied the MAPK/ERK pathway.   
	
%	I am willing to engage in complex problems in biology. My general interest is in biophysical models. Moreover, I am into using machine learning as a solver for biological PDE models. I have outstanding programming skills because I want to see the result of my work in action. I actively research my project from various angles and am eager to find creative ways to overcome challenges in the way. Hence, I think I am suited for your project.
	